\documentclass[a4paper]{article} % this is used for comments
\usepackage[utf8]{inputenc}
%%% Дополнительная работа с математикой
\usepackage{amsmath,amsfonts,amssymb,amsthm,mathtools} % AMS
\usepackage{icomma} % "Умная" запятая: $0,2$ --- число, $0, 2$ --- перечисление
\usepackage[english,russian]{babel}

%% Номера формул
\mathtoolsset{showonlyrefs=true} % Показывать номера только у тех формул, на которые есть \eqref{} в тексте.

%% Шрифты
\usepackage{euscript}	 % Шрифт Евклид
\usepackage{mathrsfs} % Красивый матшрифт

\usepackage{physics}

%% Свои команды
\DeclareMathOperator{\sgn}{\mathop{sgn}}

%% Перенос знаков в формулах (по Львовскому)
\newcommand*{\hm}[1]{#1\nobreak\discretionary{}
{\hbox{$\mathsurround=0pt #1$}}{}}

\DeclareMathOperator{\Lin}{\mathrm{Lin}}
\DeclareMathOperator{\Linp}{\Lin^{\perp}}
\DeclareMathOperator*\plim{plim}
%\DeclareMathOperator{\grad}{grad}
\DeclareMathOperator{\card}{card}
%\DeclareMathOperator{\sgn}{sign}
\DeclareMathOperator{\sign}{sign}

\DeclareMathOperator*{\argmin}{arg\,min}
\DeclareMathOperator*{\argmax}{arg\,max}
\DeclareMathOperator*{\amn}{arg\,min}
\DeclareMathOperator*{\amx}{arg\,max}
\DeclareMathOperator{\cov}{Cov}
\DeclareMathOperator{\Var}{Var}
\DeclareMathOperator{\Cov}{Cov}
\DeclareMathOperator{\Corr}{Corr}
\DeclareMathOperator{\pCorr}{pCorr}
\DeclareMathOperator{\E}{\mathbb{E}}
\let\P\relax
\DeclareMathOperator{\P}{\mathbb{P}}

\newcommand{\cN}{\mathcal{N}}
\newcommand{\cU}{\mathcal{U}}
\newcommand{\cBinom}{\mathcal{Binom}}
\newcommand{\cBin}{\cBinom}
\newcommand{\cPois}{\mathcal{Pois}}
\newcommand{\cBeta}{\mathcal{Beta}}
\newcommand{\cGamma}{\mathcal{Gamma}}

\newcommand \R{\mathbb{R}}
\newcommand \N{\mathbb{N}}
\newcommand \Z{\mathbb{Z}}

\newcommand{\dx}[1]{\,\mathrm{d}#1} % для интеграла: маленький отступ и прямая d
\newcommand{\ind}[1]{\mathbbm{1}_{\{#1\}}} % Индикатор события
%\renewcommand{\to}{\rightarrow}
\newcommand{\eqdef}{\mathrel{\stackrel{\rm def}=}}
\newcommand{\iid}{\mathrel{\stackrel{\rm i.\,i.\,d.}\sim}}
\newcommand{\const}{\mathrm{const}}

% вместо горизонтальной делаем косую черточку в нестрогих неравенствах
\renewcommand{\le}{\leqslant}
\renewcommand{\ge}{\geqslant}
\renewcommand{\leq}{\leqslant}
\renewcommand{\geq}{\geqslant}

\title{Решение промежуточного экзамена по теории вероятностей 2019-2020, вариант $\rho$}
\date{}

\begin{document}

\maketitle

% Вставить ответы!!!
\textbf{Ответы:}
CEAAC BBDAC CACEA AECBD AABAD EDBAE


\begin{enumerate}

    %Задача 1
    \item
    Из условия известно, что вероятность того, что Месси забьёт гол при ударе по воротам:
    \[
    \P(\text{гола}) = 0.95
    \]
    
    Посчитаем вероятность того, что Месси забьёт ровно три мяча за пять ударов.
    Вероятность того, что Месси промахнётся при ударе:
    \[
    \P(\text{промаха}) = 1 - \P(\text{гола}) = 1 - 0.95 = 0.05
    \]
    
    В случае, если Месси забивает ровно три гола, то в оставшихся двух случаях он промахнётся.
    Тогда искомая вероятность события $A$ (ровно три гола из пяти ударов):
    \[
    \P(A) = C_{5}^{3}0.95^{3}\cdot0.05^2
    \]
    
    \textbf{Ответ:} C.
    
    
    %Задача 2
    \item
    Ковариационная матрица вектора $X = (X_{1}, X_{2})$ по определению имеет вид:
    \[
    \begin{pmatrix}
    	\Var(X_{1}) & \Cov(X_{1}, X_{2}) \\
    	\Cov(X_{1}, X_{2}) & \Var(X_{2})
    \end{pmatrix}
    \]
    
    Из условия известна следующая матрица ковариации:
    \[
    \begin{pmatrix}
    10 & 3 \\
    3 & 8
    \end{pmatrix}
    \]
    
    Дисперсия разности элементов вектора $X$ может быть найдена по формуле:
    \[
    \Var(X_{1} - X_{2}) = \Var(X_{1}) + \Var(X_{2}) - 2\Cov(X_{1}, Y_{2}) = 10 + 8 - 2\cdot 3 = 12
    \]
    
    \textbf{Ответ:} E.
    
    
    %Задача 3
    \item
    Ковариация двух случайных величин может быть рассчитана по следующей формуле:
    \[
    \Cov(X,Y) = \E(XY) - \E(X)\E(Y)
    \]
    
    Так как случайная величина $X$ (сумма очков, выпавших на первых семи кубиках) и $Y$ (сумма очков, выпавших на следующих восьми кубиках) — независимые случайные величины, то $\E(X)(Y) = \E(X)\E(Y)$.
    Следовательно, получаем следующее:
    \[
    \Cov(X,Y) = \E(XY) - \E(X)\E(Y) = \E(X)\E(Y) - \E(X)\E(Y) = 0
    \]
    
    \textbf{Ответ:} A.
    
    
    %Задача 4
    \item 
    Так как из четырёх монет две — с «орлами» на обеих сторонах, то вероятность того, что такая монета будет взята, равняется $\frac{2}{4} = \frac{1}{2}$.
    В случае, если это произошло, из двух бросков гарантированно оба будут соответствовать исходу «выпал орёл».
    Вероятность того, что взятая монета будет «правильной», равняется $\frac{2}{4} = \frac{1}{2}$.
    В таком случае вероятность того, что оба раза выпадет «орёл», равняется $\frac{1}{2}\cdot\frac{1}{2}$.
    Тогда искомая вероятность события $A$ (после двух бросков случайной монеты оба раза выпал «орёл») равна:
     \[
     \P(A) = \frac{1}{2} + \frac{1}{2} \cdot \frac{1}{2} \cdot \frac{1}{2} = \frac{5}{8}
     \]
    
    \textbf{Ответ:} A.
    
    
    %Задача 5
    \item
    Из сходимости последовательности случайных величин $\xi_{1}$, $\xi+{2}$, $\ldots$ к невырожденной случайной величине $\xi$ по вероятности всегда следует сходимость этой последовательности к $\xi$ и по распределению.
    
    \textbf{Ответ:} C.
    
    
    %Задача 6
    \item
    Известно, что функция плотности для случайной величины $\xi$, распределённой равномерно на отрезке $[a; b]$ имеет вид:
    \[
    f_{\xi}(x) =
    \begin{cases}
    \frac{1}{b - a}, & x \in [a; b], \\
    0, & x \notin [a; b].
    \end{cases}
    \]
    
    Найдём значение функции плотности для случайной величины $\xi$, распределённой равномерно на отрезке $[0; 2]$, в точке $x = 0.25$:
    \[
    f_{\xi}(0.25) = \frac{1}{2 - 0} = \frac{1}{2}
    \]
    
    \textbf{Ответ:} B.
    

    %Задача 7
    \item
    Из условия известно, что случайные величины $X$ и $Y$ распределены одинаково и равновероятно принимают только два значения: $-1$ и $1$.
    При этом известно:
    \[
    \P(Y = 1 \mid X = x) = 0.4 + 0.3x
    \]
    
    Тогда вероятность $\P(Y = -1 \mid X = x)$ может быть рассчитана как:
    \[
    \P(Y = -1 \mid X = x)  = 1 - \P(Y = 1 \mid X = x) = 1 - (0.4 + 0.3x) = 0.6 - 0.3x
    \]
    
    Чтобы найти математическое ожидание $\E(Y \mid X = 1)$, нам понадобятся значения вероятностей $\P(Y = 1 \mid X = 1)$ и $\P(Y = -1 \mid X = 1)$:
    \[
    \P(Y = 1 \mid X = x) = 0.4 + 0.3 \cdot 1 = 0.7
    \]
    
    \[
    \P(Y = -1 \mid X = x)  = 0.6 - 0.3 \cdot 1 = 0.3
    \]
    
    Тогда значение математического ожидания можно посчитать как:
    \[
    \E(Y \mid X = 1) = 1 \cdot 0.7 - 1 \cdot 0.3 = 0.4 
    \]
    
    \textbf{Ответ:} B.
    
    
    %Задача 8
    \item
    Известно, что корреляция случайных величин $X$ и $Y$ может быть найдена по формуле:
    \[
    \Corr(X, Y) = \frac{\cov(X, Y)}{\sqrt{\var(X) \cdot \var(Y)}}
    \]
    
    Тогда корреляция случайных величин $3 - X$ и $4Y - 5$ равна:
    \begin{align*}
    \Corr(3 - X, 4Y - 5) = \frac{\cov(3 - X, 4Y - 5)}{\sqrt{\var(3 - X) \cdot \var(4Y - 5)}} = \\
    = \frac{\cov(3, 4Y) + \cov(3, -5) + \cov(-X, 4Y) + \cov(-X, -5)}{\sqrt{\var(X) \cdot 16 \cdot \var(Y)}} = \\
    = \frac{\cov(-X, 4Y)}{\sqrt{1 \cdot 16 \cdot 9}} = \frac{\cov(-X, 4Y)}{12} = \frac{-4\cov(X, Y)}{12} = \frac{-3}{-3} = 1
    \end{align*}
    
    \textbf{Ответ:} D.
    
    
    %Задача 9
    \item
    
    \[
    \E(\xi(\xi + \lambda)) = \E(\xi^{2}) + \lambda\E(\xi) = \Var(\xi) + \E^{2}(\xi) + \lambda\E(\xi)
    \]
    
    Тогда при $\E(\xi) = \frac{1}{\lambda}$ и $\Var(\xi) = \frac{1}{\lambda^{2}}$ по свойству экспоненциального распределения получается:
    \[
    \E(\xi(\xi + \lambda)) = \frac{1}{\lambda^{2}} + \left( \frac{1}{\lambda} \right)^{2} + \lambda \cdot \frac{1}{\lambda} = \frac{1}{\lambda^{2}} + \frac{1}{\lambda^{2}} + \frac{\lambda}{\lambda} = \frac{2}{\lambda^{2}} + 1
    \]
    
    \textbf{Ответ:} A.
    
    
    %Задача 10
    \item
    
    Если функция плотности случайной величины $X$ равна
    $
    f(x) =
    \begin{cases}
    4x^{3}, & \text{при} 0 < x < 1, \\
    0, & \text{иначе}
    \end{cases}
    $,
    то плотность величины $Y = \ln(\frac{1}{X})$ может быть найдена следующим образом:
    \[
    g_{Y}(y) = 
    \begin{cases}
    4\exp(-4y), & \text{если} y > 0, \\
    0, \text{иначе}.
    \end{cases}
    \]
    
    \textbf{Ответ:} C.
    
    
    %Задача 11
    \item
    Вероятность того, что на кубике при броске выпадет «5» равна:
    \[
    \P(\text{выпало «5»}) = \frac{1}{6}
    \]
    
    Тогда наиболее вероятное количество выпадений «5»:
    
   \[
   2020 \cdot \frac{1}{6} \approx 336
   \]
    
    \textbf{Ответ:} C.
    
    
    %Задача 12
    \item
    Так как из четырёх монет две — с «орлами» на обеих сторонах, то вероятность того, что такая монета будет взята, равняется $\frac{2}{4} = \frac{1}{2}$.
    В случае, если это произошло, из двух бросков гарантированно оба будут соответствовать исходу «выпал орёл».
    Вероятность того, что взятая монета будет «правильной», равняется $\frac{2}{4} = \frac{1}{2}$.
    В таком случае вероятность того, что оба раза выпадет «орёл», равняется $\frac{1}{2}\cdot\frac{1}{2}$.
    Тогда искомая вероятность события $A$, что при выпадении «орла» была выбрана «неправильная» монетка:
    \[
    \P(A) = \frac{\P(\text{«неправильная» монетка})}{\P(\text{выпал «орёл»})} = \frac{\frac{1}{2}}{\frac{1}{2} + \frac{1}{2}\cdot\frac{1}{2}} = \frac{2}{3}
    \]

    \textbf{Ответ:} A.
    
    
    %Задача 13
    \item
    Из условия известно, что $\P(A) = 0.5$ и $\P(B \mid A) = 0.8$.
    То есть:
    \[
    \P(B \mid A) = \frac{\P(AB)}{\P(A)} = 0.8
    \]
    
    Тогда из этого следует, что:
    \[
    \P(AB) = 0.8 \cdot \P(A) = 0.8 \cdot 0.5 = 0.4
    \]
    
    Также найдём:
    \[
    \P(AB + A\bar{B}) = \P(A(B + \bar{B})) = \P(A \cdot 1) = \P(A) = 0.5
    \]
    
    Тогда получаем:
    \[
    P(A\bar{B}) = \P(AB + A\bar{B}) - \P(AB) = 0.5 - 0.4 = 0.1
    \]
    
    \textbf{Ответ:} C.
    
    
    %Задача 14
    \item
    Обозначим годы ожидания письма как случайную величину $\xi$, распределённую экспоненциально с параметром $\lambda = \frac{1}{3}$.
    Известно, что функция распределения имеет вид:
    \[
    F_{\xi}(x)=
    \begin{cases}
    1 - \exp^{-\lambda x}, & \text{если} x \ge 0, \\
    0, & \text{иначе}.
    \end{cases}
    \]
    
    Тогда найдём искомую вероятность как:
    \[
    F_{\xi}(18 - 6 - 9) = 1 - \exp^{-\frac{1}{3} \cdot (18 - 6 - 9)} = -\exp(-1)
    \]
    
    \textbf{Ответ:} E.
    
    
    %Задача 15
    \item
    Если случайным образом выбираются семьи с двумя детьми и обозначаются события: $A$ — «в семье старший ребёнок — мальчик», $B$ — «в семье только один из детей — мальчик» и $C$ — «в семье хотя бы один из детей — мальчик», то события $A$ и $B$ независимы, $A$ и $C$ — зависимы, $B$ и $C$ — зависимы.
    
    \textbf{Ответ:} A.
    
    
    %Задача 16
    \item
    \begin{align*}
    \E(X \mid Y = -2) = 3 \cdot \P(X = 3 \mid Y = -2) + 6 \cdot \P(X = 6 \mid Y = -2) = \\
    = 3 \cdot \frac{0.1}{0.5} + 6 \cdot \frac{0.4}{0.5} = 0.6 + 4.8 = 5.4
    \end{align*}
    
    \textbf{Ответ:} A.
    
    
    %Задача 17
    \item
    Из условия известно, что совместная функция плотности пары $X$ и $Y$ имеет вид:
    \[
    f(x, y) = 
    \begin{cases}
    \frac{(x + y)}{3}, & \text{если} x \in [0; 1], y \in [0; 2], \\
    0, & \text{иначе}.
    \end{cases}
    \]
    
    Тогда найдём вероятность $\P(X < 0.5, Y < 1.5)$:
    \begin{align*}
    \P(X < 0.5, Y < 1.5) = \\ 
    = \frac{1}{3} \int_{0}^{\frac{1}{2}}\int_{0}^{\frac{3}{2}} (x + y) \,dx\,dy = \frac{1}{3} \int_{0}^{\frac{1}{2}} xy + \frac{y^{2}}{2} \,dx \left.\begin{matrix} & \\ & \end{matrix}\right|_{0}^{\frac{3}{2}} = \\
    = \frac{1}{3} \int_{0}^{\frac{1}{2}} 1.5x + 1.125x \,dx = \frac{1}{3}\left(\frac{1.5x^{2}}{2} + 1.125x \right) \left.\begin{matrix} & \\ & \end{matrix}\right|_{0}^{\frac{1}{2}} = \\
    = \frac{1}{4}
    \end{align*}
    
    \textbf{Ответ:} E.
    
    
    %Задача 18
    \item
    Из условия известно, что совместная функция плотности пары $X$ и $Y$ имеет вид:
    \[
    f(x, y) = 
    \begin{cases}
    cx^{3}y^{2}, & \text{если} x \in [0; 1], y \in [0; 1], \\
    0, & \text{иначе}.
    \end{cases}
    \]
    
    Найдём значение $c$.
    По определению:
    \[
    \int_{0}^{1}\int_{0}^{1} cx^{3}y^{2} \,dx\,dy = 1
    \]
    
    То есть:
    \begin{align*}
    	\int_{0}^{1}\int_{0}^{1} cx^{3}y^{2} \,dx\,dy = \int_{0}^{1} \frac{cx^{4}y^{2}}{4} \left.\begin{matrix} & \\ & \end{matrix}\right|_{0}^{1} \,dy = \\
    	= \int_{0}^{1} \frac{cy^{2}}{4} \,dy = \frac{cy^{3}}{12} \left.\begin{matrix} & \\ & \end{matrix}\right|_{0}^{1} = \\
    	= \frac{c}{12}
    \end{align*}
    
    Тогда $c = 12$.

    \textbf{Ответ:} C.
    
    
    %Задача 19
    \item
    Обозначим энтропию, совместную энтропию и условную энтропию с помощью $H(X)$, $H(X, Y)$ и $H(X \mid Y)$ соответственно.
    Можно утверждать, что дискретные случайные величины $X$ и $Y$ независимы, если $H(X, Y) = H(X \mid Y) + H(Y \mid X)$.
    
    \textbf{Ответ:} B.
    
    
    %Задача 20
    \item
    Из условия известно, что совместная функция плотности пары $X$ и $Y$ имеет вид:
    \[
    f(x, y) = 
    \begin{cases}
    cx^{3}y^{2}, & \text{если} x \in [0; 1], y \in [0; 1], \\
    0, & \text{иначе}.
    \end{cases}
    \]
    
    Тогда условная функция плотности $f_{X \mid Y = 0.5}(x)$ равна:
    \[
    f_{X \mid Y = 0.5}(x) =
    \begin{cases}
    4x^{3}, & \text{если} x \in [0; 1], \\
    0, & \text{иначе}.
    \end{cases}
    \]
    
    \textbf{Ответ:} D.
    
    
    %Задача 21
    \item
    По центральной предельной теореме $\frac{\xi_{1}^{2019} + \ldots + \xi_{n}^{2019}}{n}$ при $n$, стремящемся к бесконечности, стремится к математическому ожиданию $\E(\xi_{i})$.
    Так как случайная величина $\xi_{i}$ принимает значения от 0 до 1, а возведение в степень не меняет этих значений, то предел остаётся неизменным.
    Тогда найдём значение $\E(\xi_{i})$:
    \[
    \E(\xi_{i}) = 0.6 \cdot 1 + 0.4 \cdot 0 = 0.6
    \]

    \textbf{Ответ:} A.
    
    %Задача 22
    \item
    Из условия известно, что случайная величина $X$ имеет функцию плотности $f_{X}(x) = 2x$ на отрезке $[0; 1]$.
    Тогда математическое ожидание $\E\left(\frac{1}{X} \right)$ можно найти как:
    \[
    \E\left(\frac{1}{X} \right) = \int_{0}^{1} \frac{1}{x} \cdot 2x \,dx = 2x \left.\begin{matrix} & \\ & \end{matrix}\right|_{0}^{1} = 2
    \]
    
    \textbf{Ответ:} A.
    
    
    %Задача 23
    \item
    Из условия известно, что случайные величины $X$ и $Z$ независимы, случайная величина $X$ имеет стандартное нормальное распределение, а случайная величина $Z$ равномерно принимает значения $-1$ и $1$.
    Тогда произведение $(X + 1)Z$ имеет нечётное число скачков в функции распределения.
    
    \textbf{Ответ:} B.
    
    
    %Задача 24
    \item
    Вероятность того, что произойдёт хотя бы один сбой, найдём как:
    \[
    \P(x \ge 1) = 1 - \P(x = 0) = 1 - 1 \cdot \exp(-6) = 1 - \exp(-6)
    \]
    
    \textbf{Ответ:} A.
    
    
    %Задача 25
    \item
    Из условия известно, что математическое ожидание случайной величины $X$ равняется нулю.
    Неравенство Чебышева имеет вид:
    \[
    \P(\abs{Y - \E(Y)} \ge b) \le \frac{\Var(Y)}{b^{2}}
    \]
    
    Тогда:
    \[
    \P(\abs{X} \le 5\sqrt{\Var(X)}) = 1 - \P(\abs{X} \ge 5\sqrt{\Var(X)}) \le \frac{\Var(X)}{5^{2}\sqrt{\Var(X)}^{2}} = \frac{1}{25}
    \]
    
    Тогда вероятность лежит в интервале $[1 - 0.04; 1] = [0.96; 1]$.
    
    \textbf{Ответ:} D.
    
    
    %Задача 26
    \item
    Из условия известно, что функция распределения случайной величины $X$ имеет вид
    \[
    F_{X}(x) =
    \begin{cases}
    0, & \text{если} x < 0, \\
    cx^{2}, & \text{если} x \in [0; 1], \\
    1, & \text{если} x > 1.
    \end{cases}
    \]
    
    Найдём значение $c$:
    \[
    f_{X}(x) = F_{X}^{'}(x) = 2x
    \]
    
    Тогда:
    \[
    \int_{0}^{1} 2x \cdot c \,dx = \frac{2x^{2}}{2} c = 1 = c
    \]
    
    Тогда:
    \[
    \E(x^{2}) = \int_{0}^{1} x^{2} \cdot 2x \,dx = \int_{0}^{1} 2x^{3} \,dx = 2 \frac{x^{4}}{4} \left.\begin{matrix} & \\ & \end{matrix}\right|_{0}^{1} = \frac{1}{2}
    \]
    
    \textbf{Ответ:} E.
    
    
    %Задача 27
    \item
    Вероятность $\P(X^{2} \le 49)$ для случайной величины $X$ с известным математическим ожиданием $\E(X) = 1$ обязательно попадёт в интервал $\left[0; \frac{1}{7} \right]$
    
    \textbf{Ответ:} D.
    
    
    %Задача 28
    \item
    Обозначим за $X$ количество съедаемого Винни-Пухом мёда.
    Из условия известно, что функция распределения случайной величины $X$ имеет вид
    \[
    F_{X}(x) =
    \begin{cases}
    0, & \text{если} x < 0, \\
    cx^{2}, & \text{если} x \in [0; 1], \\
    1, & \text{если} x > 1.
    \end{cases}
    \]
    
    Найдём значение $c$:
    \[
    f_{X}(x) = F_{X}^{'}(x) = 2x
    \]
    
    Тогда:
    \[
    \int_{0}^{1} 2x \cdot c \,dx = \frac{2x^{2}}{2} c = 1 = c
    \]
    
    Теперь найдём искомую вероятность:
    \[
    \int_{\frac{2}{3}}^{1} x^{2} \,dx = x^{2} \left.\begin{matrix} & \\ & \end{matrix}\right|_{\frac{2}{3}}^{1} = 1 - \frac{4}{9} = \frac{5}{9}
    \]
    
    \textbf{Ответ:} B.
    
    
    %Задача 29
    \item
    С помощью свойств дисперсии найдём:
    \begin{align*}
    \Var(2X - Y) = 4\Var(X) + \Var(Y) + 2\Cov(2X, -Y) = \\
    = 4 \cdot 4 + 1 \cdot 9 + 2 \cdot (-2)\Cov(X, Y) = 16 + 9 - 4\Cov(X, Y) = \\
    = 16 + 9 - 4 \cdot (-3) = 25 +12 = 37
    \end{align*}
    
    
    \textbf{Ответ:} A.
    
    
    %Задача 30
    \item
    Найдём значение $\E(\bar{X}_{100})$:
    \begin{align*}
    \E(\bar{X}_{100}) = \E\left(\frac{\sum_{i = 1}^{n} X_{i}}{n} \right)  = \frac{1}{n} \E\left(\sum_{i = 1}^{n} X_{i} \right) = \\
    = \frac{1}{n} \E(X_{1} + \ldots + X_{n}) = \frac{1}{n} ( \E(X_{1}) + \ldots + \E(X_{n}))  = \\
    = \frac{1}{n} \cdot n  \cdot \E(X_{1}) = \E(X_{1}) = 1
    \end{align*}
    
    Найдём значение $\Var(\bar{X}_{100})$:
    \begin{align*}
    \Var(\bar{X}_{100}) = \Var\left(\frac{\sum_{i = 1}^{n} X_{i}}{n} \right) = \frac{1}{n^{2}} \Var\left(\sum_{i = 1}^{n} X_{i} \right) = \\
    =  \frac{1}{n^{2}} (\Var(X_{1}) + \ldots + \Var(X_{n})) = \frac{1}{n^{2}} \cdot n\Var(X_{1}) = \\
    = \frac{\Var(X_{1})}{n} = \frac{49}{100} = 0.49
    \end{align*}
    
    Приведём случайную величину $\bar{X}_{100}$ к случайной величине $Z \sim \mathcal{N}(0; 1)$ и найдём искомую вероятность:
    \begin{align*}
    	\P(\bar{X}_{100} < 2) = \P\left(\frac{\bar{X}_{100} - \E(\bar{X}_{100})}{\sqrt{\Var(\bar{X}_{100})}} < \frac{2 - \E(\bar{X}_{100})}{\sqrt{\Var(\bar{X}_{100})}} \right) = \\
    	= \P\left(Z < \frac{2-1}{\sqrt{0.49}} \right) = \P\left(Z < \frac{1}{0.7} \right) = \P\left(Z < \frac{10}{7} \right) = \\
    	= \P(Z < 1.429) = 0.92
    \end{align*}
    
    \textbf{Ответ:} E.
    
    
\end{enumerate}



\end{document}